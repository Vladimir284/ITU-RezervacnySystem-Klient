\section{Testování}
Každý z členů si našel minimálně dva kandidáty, které odpovídali jeho minimálním požadavkům na otestování softwaru a následně nám pomohli zkusit otestovat intuitivnost našeho zpracování projektu.

\subsection{Administrativní pracovník}

Testování stránky administrativního pracovníka probíhalo na starší úřední osobě.
Bylo testováno, zda lze najít konkrétní záznam, aniž by se musel ptát na cestu nebo proklikávat jinými cestami.
Na první pohled byly některé tlačítka poměrně špatně viditelné.
Změna zobrazovaného dne nebyla nijak moc problematická.
Není patrné(nezobrazí se správný kurzor), že na položky v kalendáři lze klikat.
Nepochopení významu vytíženosti zaměstnanců v rámci jedné služby.
\emph{Několik drobných chyb chování. Například: Nezobrazení textu "Všichni" při otevření kalendáře z vytíženosti konkrétní služby, přes kliknutí na den v sloupcovém grafu.}

Sehnali jsme na testování jednu mladší doktorku, která se ochotně zúčastnila. První na domovské stránce se statistikami zatížení dostala za úkol najít rozvrh zaměstnance \emph{Ondřeje Fojta} a ač nikde nebyla hned možnost s nápisem Rozvrh, tak rychle sama od sebe správně zareagovala a klikna na tlačítko \emph{Zaměstnanci} z kterého se nakonec dostala k rozvrhu daného zaměstnance. Sice měla testující vnitřní pocit nejistoty a snažila se ujistit, zda to nedělá špatně, ale i bez odpovědi to zvládla najít sama. 
Dalším úkol bylo najít vytíženost jedné konkrétní služby. Jednoduše se z rozvrhu zvládla vrátit na domovskou stránku a najít konkrétní službu a oznámit její vytíženost. 
Poslední úkol byl najít detailní informace o rezervacích. Testující zašla do kalendáře, našla si rezervaci a zjistila všechny potřebné údaje. Celkově ohodnotila práci se službou kladně až na mírné technické problémy po kliknutí na prázdné políčko rozvrhu. 

Tlačítko domů nejde dobře vidět a je hodně zřejmá absence možnosti kroku zpět.
U zobrazování Vytíženosti služeb v následující dny nedává smysl zobrazovat také dny předchozí.
Popisky s nápovědou, co která akce udělá/ukáže.

\subsection{Fyzioterapeut}
Testování rozhraní pro fyzioterapeuta proběhlo na dvou osobách. První byla osoba v pokročilejším věku. Přes svůj věk má několik zkušeností s počítači. Testování proběhlo zadáváním cílů, kde si osoba, která testovala musela sama poradit, jak software k dosažení využít. Např. první osoba bylo za úkol zadáno najít v rozvrhu konkrétního pacienta v některém z nespecifikovaných následujících týdnů. Překvapivě si vůbec první osoba nevšimla šipek a myslela si, že políčko s následující osobou je vyhledávání. Nakonec na to přišla a našla daného pacienta.

Druhá mladší dospělá osoba, která testovala stejný úkol ho zvládla bez problému v překvapivě krátkém čase a stejně tak, všechny další alternativy zadání.
 
Metrikem bylo jednoznačně primárně očekávání, jak bude rozhraní použito, které se částečně splnilo. Zklamáním byl výsledek první starší osoby, ale díky němu přišli náměty na vylepšení. Např. přidání vyhledávání a mírnou změnu vzhledu položky následuje, aby tak nepřipomínala vyhledávání. 

\subsection{Pacient}
viz sekce \ref{client_testing}
